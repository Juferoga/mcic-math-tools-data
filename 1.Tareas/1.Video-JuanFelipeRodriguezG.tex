\documentclass{article}
\usepackage[utf8]{inputenc}
\usepackage[spanish]{babel}
\usepackage{amsmath}

\title{Reseña Video The Strange Math That Predicts (Almost) Anything}
\author{Juan Felipe Rodríguez G. (20261595004)}
\date{\today}

\begin{document}
    \maketitle
    \section*{Reseña 1 del vídeo}
    En el video \textit{The Strange Math That Predicts (Almost) Anything}, el científico australiano-canadiense Derek Muller presenta una perspectiva interesante mostrando la evolución de las cadenas de Markov. A través del video enseña cómo la aplicación de estas matemáticas incide en la predicción de eventos, fenómenos tanto naturales como antropogénicos.

    Uno de los aspectos que resulta más destacable es la forma en que un concepto puede tener implicaciones en tantas ramas del conocimiento y de la misma vida humana. Por ejemplo, el concepto de probabilidad se inmiscuye en varios ámbitos, tales como: el funcionamiento de la bomba atómica que dio origen a la pérdida de cientos de miles de vidas humanas, la aplicación en la predicción de caracteres en modelos de lenguaje de gran escala y el uso de eventos tanto dependientes como independientes.

    Tomando la idea de Markov, en algunas ocasiones prima la rigurosidad científica matemática y en su contraparte la creatividad. Esta última juega un papel importante en la historia y el desarrollo humano. Esto permite la gestación de ideas innovadoras, no limitándose a la matemática como mero artificio, sino como la herramienta capaz de construir y modelar eventos.

    El hecho de poder obtener una predicción acertada para un usuario resulta fascinante, puesto que el usuario ve el funcionamiento de esta máquina como arte de magia, esto evoca la historia del ajedrecista turco, autómata aparentemente capaz de comunicarse y efectuar actos por sí mismo, el cual resultó ser un fraude, manipulado por un humano desde su interior.

    \section*{Reseña 2 del vídeo}
    
    Al realizar la segunda visualización del vídeo, mi atención se dirigió hacia la pragmática de los modelos estocásticos y su alineación directa con los objetivos del syllabus de la asignatura \textit{Herramientas Matemáticas para el Manejo de la Información}. Se hace evidente que los fundamentos de probabilidad y estadística, tal como se exponen en textos guía como el de Walpole et al. \cite{walpole2017}, no son meras abstracciones, sino la base operativa para modelar la incertidumbre en sistemas de ingeniería.

    Y luego, tras ver el syllabus, vi que el enfoque del curso hacia el análisis de datos y la inferencia estadística cobra un sentido tangible al contrastarlo con la historia de las cadenas de Markov. Comprendí que las ``herramientas'' a las que se refiere el curso son precisamente estos artificios matemáticos que nos permiten simplificar la realidad. Al igual que en la estadística multivariante mencionada por Peña \cite{pena2001}, el objetivo final es reducir la dimensionalidad y complejidad de los datos para hacerlos manejables.

    Específicamente, cuando en el vídeo Muller decía que para muchos sistemas complejos ``puedes ignorar casi todo el pasado y solo observar el estado actual'', se resalta la propiedad de \textit{falta de memoria} (Markoviana). Esta simplificación es vital en las Ciencias de la Información, pues transforma problemas intratables (como indexar todo internet o predecir el clima considerando toda la historia) en problemas computables basados en estados presentes y probabilidades de transición.

    Esta relación me permitió comprender que la asignatura no trata solo de la exactitud numérica, sino de la capacidad de modelar sistemas dinámicos. Las cadenas de Markov, aplicadas desde el algoritmo PageRank de Google hasta los LLMs modernos, ejemplifican perfectamente cómo una base matemática sólida en probabilidad permite gestionar y predecir el comportamiento de grandes volúmenes de información, validando la pertinencia de los temas propuestos en el plan de estudios.

    \begin{thebibliography}{}

    \bibitem{walpole2017}
    Walpole, R. E., Myers, R. H., Myers, S. L., \& Ye, K. (2017).
    \textit{Probability and Statistics for Engineers and Scientists} (9th ed.).
    Pearson.

    \bibitem{pena2001}
    Peña, D. (2001).
    \textit{Fundamentos de Estadística}.
    Alianza Editorial.

    \end{thebibliography}
    
\end{document}
