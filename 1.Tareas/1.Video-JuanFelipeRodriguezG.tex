\documentclass{article}
\usepackage[utf8]{inputenc}
\usepackage[spanish]{babel}
\usepackage{amsmath}

\title{Reseña Video The Strange Math That Predicts (Almost) Anything}
\author{Juan Felipe Rodríguez G.}
\date{\today}

\begin{document}
    \maketitle
    \section*{Reseña 1 del vídeo}
    En el video \textit{The Strange Math That Predicts (Almost) Anything}, el científico australiano-canadiense Derek Muller presenta una perspectiva interesante mostrando la evolución de las cadenas de Markov. A través del video enseña cómo la aplicación de estas matemáticas incide en la predicción de eventos, fenómenos tanto naturales como antropogénicos.

    Uno de los aspectos que resulta más destacable es la forma en que un concepto puede tener implicaciones en tantas ramas del conocimiento y de la misma vida humana. Por ejemplo, el concepto de probabilidad se inmiscuye en varios ámbitos, tales como: el funcionamiento de la bomba atómica que dio origen a la pérdida de cientos de miles de vidas humanas, la aplicación en la predicción de caracteres en modelos de lenguaje de gran escala y el uso de eventos tanto dependientes como independientes.

    Tomando la idea de Markov, en algunas ocasiones prima la rigurosidad científica matemática y en su contraparte la creatividad. Esta última juega un papel importante en la historia y el desarrollo humano. Esto permite la gestación de ideas innovadoras, no limitándose a la matemática como mero artificio, sino como la herramienta capaz de construir y modelar eventos.

    El hecho de poder obtener una predicción acertada para un usuario resulta fascinante, puesto que el usuario ve el funcionamiento de esta máquina como arte de magia, esto evoca la historia del ajedrecista turco, autómata aparentemente capaz de comunicarse y efectuar actos por sí mismo, el cual resultó ser un fraude, manipulado por un humano desde su interior.

    \section*{Reseña 2 del vídeo}
\end{document}
